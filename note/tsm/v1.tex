\documentclass[UTF8,a5paper]{ctexart}
\special{papersize=148mm,210mm}
\title{羊皮卷之一:今天,我开始新的生活}
\date{\today}
\begin{document}
\fangsong
\large
\maketitle

今天,我开始新的生活。

今天,我爬出满是失败创伤的老茧。

今天,我重新来到这个世上,我出生在葡萄园中,园内的葡萄任人享用。

今天,我要从最高最密的藤上摘下智慧的果实,这葡萄藤是好几代前的智者种下的。

今天,我要品尝葡萄的美味,还要吞下每一颗成功的种子,让新生命在我心里萌芽。

我选择的道路充满机遇,也有辛酸与绝望。失败的同伴数不胜数,叠在一起,比金字塔还高。

然而,我不会像他们一样失败,因为我手中持有航海图,可以领我越过汹涌的大海,抵达梦中的彼岸。

失败不再是我奋斗的代价。它和痛苦都将从我的生命中消失。失败和我,就像水火一样,互不相容。我不再像过去一样接受它们。我要在智慧的指引下,走出失败的阴影,步入富足、健康、快乐的乐园,这些都超出了我以往的梦想。

我要是能长生不老,就可以学到一切,但我不能永生,所以,在有限的人生里,我必须学会忍耐的艺术,因为大自然的行为一向是从容不迫的。造物主创造树中之王橄摊树需要一百年的时间,而洋葱经过短短的九个星期就会枯老。我不留恋从前那种洋葱式的生活,我要成为万树之王——橄榄树,成为现实生活中最伟大的推销员。

怎么可能?我既没有渊博的知识,又没有丰富的经验,况且,我曾一度跌入愚昧与自怜的深渊。答案很简单。我不会让所谓的知识或者经验妨碍我的行程。造物主已经赐予我足够的知识和本能,这份天赋是其它生物望尘莫及的。经验的价值往往被高估了,人老的时候开口讲的多是糊涂话。

说实在的,经验确实能教给我们很多东西,只是这需要花费太长的时间。等到人们获得智慧的时候,其价值已随着时间的消逝而减少了。结果往往是这样,经验丰富了,人也余生无多。经验和时尚有关,适合某一时代的行为,并不意味着在今天仍然行得通。

只有原则是持久的,而我现在正拥有了这些原则。这些可以指引我走向成功的原则全写在这几张羊皮卷里。它教我如何避免失败,而不只是获得成功,因为成功更是一种精神状态。人们对于成功的定义,见仁见智,而失败却往往只有一种解释:\textbf{失败就是一个人没能达到他的人生目标,不论这些目标是什么。}

事实上,成功与失败的最大分野,来自不同的习惯。好习惯是开启成功的钥匙,坏习惯则是一扇向失败敞开的门。因此,我首先要做的便是养成良好的习惯,全心全意去实行。

小时候。我常会感情用事,长大成人了,我要用良好的习惯代替一时的冲动。我的自由意志屈服于多年养成的恶习,它们威胁着我的前途。我的行为受到品味、情感、偏见、欲望、爱、恐惧、环境和习惯的影响,其中最厉害的就是习惯。因此,如果我必须受习惯支配的话,那就让我受好习惯的支配。那些坏习惯必须戒除,我要在新的田地里播种好的种子。

我要养成良好的习惯,全心全意去实行。

这不是轻而易举的事情,要怎样才能做到呢?靠这些羊皮卷就能做到。因为每一卷里都写着一个原则,可以摒除一项坏习惯,换取一个好习惯,使人进步,走向成功。这也是自然法则之一,只有一种习惯才能抑制另一种习惯。所以,为了走好我选择的道路,我必须养成的第一个习惯是:

\textbf{每张羊皮卷用三十天的时间阅读,然后再进人下一卷。}

清晨即起,默默诵读;午饭之后,再次默读;夜晚睡前,高声朗读。

第二天的情形完全一样。这样重复三十天后,就可以打开下一卷了。每一卷都依照同样的方法读上三十天,久而久之,它们就成为一种习惯了。

这些习惯有什么好处呢?这里隐含着人类成功的秘诀。当我每天重复这些话的时候,它们成了我精神活动的一部分,更重要的是,它们渗入我的心灵。那是个神秘的世界,永不静止,创造梦境,在不知不觉中影响我的行为。

当这些羊皮卷上的文字,被我奇妙的心灵完全吸收之后,我每天都会充满活力地醒来。我从来没有这样精力充沛过。我更有活力,更有热情,要向世界挑战的欲望克服了一切恐惧与不安。在这个充满争斗和悲伤的世界里,我竟然比以前更快活。最后,我会发现自己有了应付一切情况的办法。不久,这些办法就能运用自如。因为,任何方法,只要多练习,就会变得简单易行。

经过多次重复,一种看似复杂的行为就变得轻而易举,实行起来,就会有无限的乐趣,有了乐趣,出于人之天性,我就更乐意常去实行。于是,一种好的习惯便诞生了。习惯成为自然。既是一种好的习惯,也就是我的意愿。

今天,我天始新的生活。

\uwave{我郑重地发誓,绝不让任何事情妨碍我新生命的成长。}在阅读这些羊皮卷的时候,我绝不浪费一天的时间,因为时光一去不返,失去的日子是无法弥补的。我也绝不打破每天阅读的习惯。事实上,每天在这些新习惯上花费的少许时间,相对于可能获得的快乐与成功而言,只是微不足道的代价。

当我阅读羊皮卷中的字句时,绝不能因为文字的精炼而忽视内容的深沉。一瓶葡萄美酒需要千百颗果子酿制而成,果皮和渣子抛给小鸟。葡萄的智慧代代相传,有些被过滤,有些被淘汰,随风飘逝。只有纯正的真理才是永恒的。它们就精炼在我要阅读的文字中。我要依照指示,绝不浪费,饮下成功的种子。

今天,我的老茧化为尘埃。我在人群中昂首阔步,不会有人认出我来,因为我不再是过去的自己、我已拥有新的生命。
\end{document}
